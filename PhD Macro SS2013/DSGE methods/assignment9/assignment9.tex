%% This document created by Scientific Word (R)
%% Version 1.1


\documentstyle[12pt,thmsa,sw20lart]{article}
%%%%%%%%%%%%%%%%%%%%%%%%%%%%%%%%%%%%%%%%%%%%%%%%%%%%%%%%%%%%%%%%%%%%%%%%%%%%%%%%%%%%%%%%%%%%%%%%%%%%%%%%%%%%%%%%%%%%%%%%%%%%%%%%%%%%%%%%%%%%%%%%%%%%%%%%%%%%%%%%%%%%%%%%%%%%%%%%%%%%%%%%%%%%%%%%%%%%%%%%%%%%%%%%%%%%%%%%%%%%%%%%%%%%%%%%%%%%%%%%%%%%%%%%%%%%
%TCIDATA{TCIstyle=Article/ART4.LAT,lart,article}

%TCIDATA{OutputFilter=LATEX.DLL}
%TCIDATA{Version=5.50.0.2953}
%TCIDATA{<META NAME="SaveForMode" CONTENT="1">}
%TCIDATA{BibliographyScheme=Manual}
%TCIDATA{Created=Thu Oct 14 19:48:26 1999}
%TCIDATA{LastRevised=Wednesday, March 21, 2012 14:32:54}
%TCIDATA{<META NAME="GraphicsSave" CONTENT="32">}
%TCIDATA{Language=American English}
%TCIDATA{CSTFile=article.cst}

\input tcilatex
\begin{document}


Christiano

\begin{center}
Tutorial on Forecasting, Output Gap Estimation, DSGE Model Estimation and
the MCMC Algorithm Using Dynare
\end{center}

\section{Clarida-Gali-Gertler Model}

Following are the equations of the Clarida-Gali-Gertler model. 
\begin{eqnarray*}
\pi _{t} &=&\beta E_{t}\pi _{t+1}+\kappa x_{t}\text{ (Calvo pricing equation)%
} \\
&& \\
x_{t} &=&-\left[ r_{t}-E_{t}\pi _{t+1}-r_{t}^{\ast }\right] +E_{t}x_{t+1}%
\text{ (intertemporal equation)} \\
&& \\
r_{t} &=&\alpha r_{t-1}+(1-\alpha )\left[ \phi _{\pi }\pi _{t}+\phi _{x}x_{t}%
\right] +u_{t}\text{ (policy rule)} \\
&& \\
r_{t}^{\ast } &=&\rho \Delta a_{t}+\frac{1}{1+\varphi }\left( 1-\lambda
\right) \tau _{t}\text{ (natural rate)} \\
&& \\
y_{t}^{\ast } &=&a_{t}-\frac{1}{1+\varphi }\tau _{t}\text{ (natural output)}
\\
&& \\
x_{t} &=&y_{t}-y_{t}^{\ast }\text{ (output gap)} \\
&&
\end{eqnarray*}%
The above equations represent the equilibrium conditions of an economy,
linearized about its steady state. In the economy, household preferences are
given by:%
\[
E_{0}\sum_{t=0}^{\infty }\left( \log C_{t}-\exp \left( \tau _{t}\right) 
\frac{N_{t}^{1+\varphi }}{1+\varphi }\right) ,\text{ }\tau _{t}=\lambda \tau
_{t-1}+\varepsilon _{t}^{\tau },\text{ }\varepsilon _{t}^{\tau }\symbol{126}%
iid, 
\]%
where $C_{t}$ denotes consumption, $\tau _{t}$ is a time $t$ preference
shock and $N_{t}$ denotes employment. The budget constraint of the household
is:%
\[
P_{t}C_{t}+B_{t+1}\leq W_{t}N_{t}+R_{t-1}B_{t}+T_{t}, 
\]%
where $T_{t}$ denotes (lump sum) taxes and profits, $P_{t}$ is the price
level, $W_{t}$ denotes the nominal wage rate and $B_{t+1}$ denotes bonds
purchased at time $t$ which deliver a non-state-contingent rate of return, $%
R_{t}$, in period $t+1.$ Competitive firms produce a homogeneous output
good, $Y_{t},$ using the following technology:%
\[
Y_{t}=\left[ \int_{0}^{1}Y_{i,t}^{\frac{\varepsilon -1}{\varepsilon }}\right]
^{\frac{\varepsilon }{\varepsilon -1}}di,\text{ }\varepsilon >1, 
\]%
where $Y_{i,t}$ denotes the $i^{th}$ intermediate good, $i\in \left(
0,1\right) .$ The competitive firms takes the price of the final output
good, $P_{t},$ and the prices of the intermediate goods, $P_{i,t},$ as given
and chooses $Y_{t}$ and $Y_{it}$ to maximize profits. This results in the
following first order condition:%
\[
Y_{i,t}=Y_{t}\left( \frac{P_{t}}{P_{i,t}}\right) ^{\varepsilon }. 
\]%
The producer of $Y_{it}$ is a monopolist which takes the above equation as
its demand curve. The $i^{th}$ intermediate good firm uses labor, $N_{i,t},$
to produce output using the following production function:%
\[
Y_{i,t}=\exp \left( a_{t}\right) N_{i,t},\text{ }\Delta a_{t}=\rho \Delta
a_{t-1}+\varepsilon _{t}^{a}, 
\]%
where $\Delta $ is the first difference operator and $\varepsilon _{t}^{a}$
is an iid shock. We refer to the time series representation of $a_{t}$ as a
`unit root' representation. The $i^{th}$ firm sets prices subject to Calvo
frictions. In particular,%
\[
P_{i,t}=\left\{ 
\begin{array}{cc}
\tilde{P}_{t} & \text{with probability }1-\theta \\ 
P_{i,t} & \text{with probability }\theta%
\end{array}%
\right. , 
\]%
where $\tilde{P}_{t}$ denotes the price chosen by the $1-\theta $ firms that
can reoptimize their price at time $t.$ The $i^{th}$ producer is competitive
in labor markets, where it pays $W_{t}\left( 1-\nu \right) $ for one unit of
labor. Here, $\nu $ represents a subsidy which has the effect of eliminating
the monopoly distortion on labor in the steady state. That is, $1-\nu
=\left( \varepsilon -1\right) /\varepsilon .$

The Ramsey equilibrium for the model is the equilibrium associated with the
optimal monetary policy. It can be shown that the Ramsey equilibrium is
characterized by zero inflation, $\pi _{t}=0,$ at each date and for each
realization of $a_{t}$ and $\tau _{t}$ and that consumption and employment
in the Ramsey equilibrium corresponds to their first best levels.\footnote{%
For a discussion, see http://faculty.wcas.northwestern.edu/\symbol{126}%
lchrist/course/optimalpolicyhandout.pdf} That is, $C_{t}$ and $N_{t}$
satisfy the resource constraint%
\[
C_{t}=\exp \left( a_{t}\right) N_{t}, 
\]
and the condition that the marginal rate of substitution between consumption
and labor equals the marginal product of labor%
\[
\frac{\text{marginal utility of leisure}}{\text{marginal utility of
consumption}}=C_{t}\exp \left( \tau _{t}\right) N_{t}^{\varphi }=\exp \left(
a_{t}\right) . 
\]%
Solving for $N_{t}:$%
\[
\log \left( N_{t}^{\ast }\right) =-\frac{\tau _{t}}{1+\varphi },\text{ }\log
\left( C_{t}^{\ast }\right) =a_{t}-\frac{\tau _{t}}{1+\varphi }, 
\]%
where $\ast $ indicates that the variable corresponds to the Ramsey
equilibrium. In the description of the model above, $y_{t}$ denotes log
output and $y_{t}^{\ast }$ denotes log output in the Ramsey equilibrium,
i.e., $\log \left( C_{t}^{\ast }\right) .$ The gross interest rate in the
Ramsey equilibrium, $R_{t}^{\ast },$ satisfies the intertemporal household
first order condition, 
\[
1=\beta E_{t}\frac{u_{c,t+1}^{\ast }}{u_{c,t}^{\ast }}\frac{R_{t}^{\ast }}{%
1+\pi _{t+1}^{\ast }}, 
\]%
where $u_{c,t}^{\ast }$ indicates the marginal utility of consumption in the
Ramsey equilibrium. Also, $\pi _{t}^{\ast }=0.$ With our utility function:%
\[
1=\beta E_{t}\frac{C_{t}^{\ast }}{C_{t+1}^{\ast }}R_{t}^{\ast }=\beta E_{t}%
\frac{R_{t}^{\ast }}{\exp \left[ \Delta a_{t+1}-\frac{\tau _{t+1}-\tau _{t}}{%
1+\varphi }\right] }=\beta E_{t}\exp \left[ \log \left( R_{t}^{\ast }\right)
-\Delta a_{t+1}+\frac{\tau _{t+1}-\tau _{t}}{1+\varphi }\right] , 
\]%
Approximately, one can `push' the expectation operator into the power of the
exponential. Doing so and taking the log of both sides, one obtains:%
\[
0=\log \beta +\log \left( R_{t}^{\ast }\right) -E_{t}\Delta a_{t+1}+E_{t}%
\frac{\tau _{t+1}-\tau _{t}}{1+\varphi }, 
\]%
or,%
\[
r_{t}^{\ast }=E_{t}\Delta a_{t+1}-E_{t}\frac{\tau _{t+1}-\tau _{t}}{%
1+\varphi }, 
\]%
where $r_{t}^{\ast }\equiv \log \left( R_{t}^{\ast }\beta \right) ,$ the log
deviation of $R_{t}^{\ast }$ from its value in the non-stochastic steady
state. The variable, $r_{t}^{\ast },$ corresponds to the `natural rate of
interest' and $y_{t}^{\ast }$ corresponds to the `natural rate of output'.

\section{Computer Exercises}

You will need the Dynare files, cggsim.mod and cggest.mod, as well as the
MATLAB m files, plots.m, analyzegap.m, suptitle.m and HPFAST.m, to do this
assignment (you can see answers in cggsimans.mod and cggestans.mod).

The HP filter is defined as follows:%
\[
\min_{\left\{ y_{t}^{T}\right\} _{t=1}}\sum_{t=1}^{T}\left(
y_{t}-y_{t}^{T}\right) ^{2}+\lambda \sum_{t=2}^{T-1}\left[ \left(
y_{t+1}^{T}-y_{t}^{T}\right) -\left( y_{t}^{T}-y_{t-1}^{T}\right) \right]
^{2} 
\]%
The parameter, $\lambda ,$ controls how `smooth' $y_{t}^{T}$ is. If $\lambda
=0,$ then $y_{t}=y_{t}^{T}.$ If $\lambda =\infty ,$ then $y_{t}^{T}$ is a
time trend (i.e., a line whose second derivative is zero). In business cycle
analysis, it is customary to use $\lambda =1600$ in studying quarterly. The
MATLAB m-file, [y\_hp,y\_hptrend]=HPFAST(y,lambda) takes $y$ as input and
puts out y\_hp=$y_{t}-y_{t}^{T},$ y\_hptrend=$y_{t}^{T}.$

This assignment explores four things: (i) the estimation of the output gap
using the HP filter and a model (ii) estimation, by Bayesian and maximum
likelihood methods, of a model, and (iii) the MCMC algorithm as a device for
approximating a posterior distribution (iv) basic economic properties of the
model.

\begin{enumerate}
\item Before turning to the econometric part of the assignment, it is useful
to study the economics of the CGG model, by seeing how the CGG economy
responds to a shock. Consider the following parameterization used:%
\begin{eqnarray*}
\beta &=&0.97,\text{ }\phi _{x}=0,\text{ }\phi _{\pi }=1.5,\text{ }\alpha =0,%
\text{ }\rho =0.2,\text{ }\lambda =0.5,\text{ }\delta =0.2,\text{ } \\
\varphi &=&1,\text{ }\theta =0.75,\text{ }\sigma _{a}=\sigma _{\tau }=0.02,%
\text{ }\sigma _{u}=0.
\end{eqnarray*}

\begin{enumerate}
\item In the case of the technology and preference shocks, use Dynare to
compute the impulse response functions of the variables to each shock. The m
file, plots.m, can be used for this purpose.

\begin{enumerate}
\item Consider the response of the economy to a technology shock and a
preference shock. In each case, indicate whether the economy over- or under-
responds to the shock, relative to their `natural' responses. What is the
economic intuition in each case?

\item Replace the time series representation of $a_{t}$ with%
\[
a_{t}=\rho a_{t-1}+\varepsilon _{t}^{a}. 
\]%
How does the response of the economy to $\varepsilon _{t}^{a}$ with this
representation compare to the response to $\varepsilon _{t}^{a}$ with the
unit root representation?
\end{enumerate}

\item Do the calculations with $\phi _{\pi }=0.99.$ What sort of message
does Dynare generate, and can you provide the economic intuition for it? (In
this case, there is `indeterminacy', which means a type of multiplicity of
equilibria...this happens whenever $\phi _{\pi }<1.)$ Provide intuition for
this result.

\item Return to the parameterization, $\phi _{\pi }=1.5.$ Now, insert $r_{t}$
into the Cavlo pricing equation. Redo the calculations and note how Dynare
reports indeterminacy again. Provide economic intuition for your result.

\item Explain why it is that when the monetary policy rule is replaced by
the $r_{t}=r_{t}^{\ast },$ the natural equilibrium (i.e., Ramsey) is a
solution to the equilibrium conditions. Explain why the natural equilibrium
is not the only solution to the equilibrium conditions (i.e., the indicated
policy rule does not support the natural equilibrium uniquely). Verify this
result computationally in Dynare.

\item Now replace the monetary policy rule with%
\[
r_{t}=r_{t}^{\ast }+\alpha \left( r_{t-1}-r_{t-1}^{\ast }\right) +(1-\alpha
) \left[ \phi _{\pi }\pi _{t}+\phi _{x}x_{t}\right] . 
\]%
Explain why the natural equilibrium is a solution to the equilibrium
conditions with this policy. Verify computationally that this policy rule
uniquely supports the natural equilibrium (in the sense of satisfying
determinacy), as long as $\phi _{\pi }$ is large enough. Provide intuition.
Conclude that the Taylor rule uniquely supports the natural equilibrium if
the natural rate of interest is included in the rule.

\item Consider the following alternative representation for the technology
shock:%
\[
a_{t}=\rho a_{t-1}+\xi _{t}^{0}+\xi _{t-1}^{1}, 
\]%
where both shocks are iid, so that the sum is iid too. Here, we assume
agents see $\xi _{t}^{0}$ at time $t$ and they see $\xi _{t-1}^{1}$ at $t-1.$
Thus, agents have advance information (or, `news') about the future
realization of a shock. Introduce this change into the code and set $\rho
=0.2.$ Verify that when there is a shock to $\xi _{t}^{1},$ inflation falls
contemporaneously and the output gap jumps. Provide intuition for this
apparently contradictory result. What happens when the natural rate of
interest is introduced in the policy rule?
\end{enumerate}

\item We now explore the MCMC algorithm and the Laplace approximation in a
simple example. Technical details about both these objects are discussed in
lecture notes.\footnote{%
See http://faculty.wcas.northwestern.edu/\symbol{126}%
lchrist/course/estimationhandout.pdf} One practical consideration not
mentioned in the notes is relevant for the case in which the pdf of interest
is of a non-negative random variable. Since the jump distribution is Normal,
a negative $x$ could be drawn (see the notes for a detailed discussion of $x$
and the `jump distribution'). As a result, we should assign a zero value to
the density of a Weibull over negative random variables when implementing
the MCMC algorithm.

Hopefully, it is apparent that the MCMC algorithm is quite simple, and can
be programmed by anyone with a relatively small exposure to MATLAB. A useful
exercise to understand how the algorithm works, is to use it to see how well
it approximates a simple known function. Thus, consider the Weibull
probability distribution function (pdf), 
\[
g\left( \theta \right) =ba^{-b}\theta ^{b-1}e^{-\left( \frac{\theta }{a}%
\right) ^{2}},\text{ }x\geq 0, 
\]%
where $a,b$ are parameters. (For an explanation of this pdf, see the MATLAB
documentation of $[g]=wblpdf(\theta ,a,b).$) Consider $a=10,$ $b=20.$ Graph
this pdf over the grid, $\left[ 7,11.5\right] ,$ with intervals $0.001$
(i.e., graph $g$ on the vertical axis, where $g=wblpdf(x,10,20),$ and $x$ on
the horizontal axis, where $x=7:.001:11.5$). Compute the mode of this pdf by
finding the element in your grid with the highest value of $g.$ Compute the
second derivative of the Weibull at the mode point numerically, using the
formula,%
\[
f^{\prime \prime }\left( x\right) =\frac{f\left( x+2\varepsilon \right)
-2f\left( x\right) +f\left( x-2\varepsilon \right) }{4\varepsilon ^{2}}, 
\]%
for $\varepsilon $ small (for example, you could set $\varepsilon =0.000001$%
.) Here, $x$ plays the role of $\theta ^{\ast }$ and $f$ plays the role of
the MATLAB function, $wblpdf.$ Set $V=-f^{\prime \prime }\left( \theta
^{\ast }\right) ^{-1}.$

Set $M=1,000$ (a very small number!) and try $k=2,4,6.$ Which implies an
acceptance rate closer to the recommended value of around 0.23? Choose the
value of $k$ that gets closest to that acceptance rate and note that the
MCMC estimate of the distribution is quite volatile. Change $M$ to 10,000.
If you have time (now, the simulations takes time) try $M=100,000.$ Note how
the MCMC estimate of the distribution is starting to smooth out. When I set $%
M=100,000$ and $k=4,$ I obtained (see the MATLAB code WeibullMCMC.m) the
following result:%
\[
\]%
\[
\FRAME{itbpF}{5.8833in}{2.8988in}{0in}{}{}{weibull.eps}{\special{language
"Scientific Word";type "GRAPHIC";maintain-aspect-ratio TRUE;display
"USEDEF";valid_file "F";width 5.8833in;height 2.8988in;depth
0in;original-width 16.6139in;original-height 8.1448in;cropleft "0";croptop
"1";cropright "1";cropbottom "0";filename 'weibull.eps';file-properties
"XNPEU";}}
\]%
\[
\]%
Note how well the MCMC approximation works. The Laplace approximation
assigns too much density near the mode, and lacks the skewness of the
Weibull. Still, for practical purposes the Laplace may be workable, at least
as a first approximation in the initial stages of a research project. This
could be verified in the early stages of the project by doing a run using
the MCMC algorithm and comparing the results with those of the Laplace
approximation. In practice, posterior distributions may not be as skewed as
the Weibull is.

\item From here on, consider the following alternative parameterization,
which is more appealing than the one in question 1 from an empirical point
of view:%
\begin{eqnarray*}
\beta &=&0.97,\text{ }\phi _{x}=0.15,\text{ }\phi _{\pi }=1.5,\text{ }\alpha
=0.8,\text{ }\rho =0.9,\text{ }\lambda =0.5,\text{ }\delta =0.2,\text{ } \\
\varphi &=&1,\text{ }\theta =0.75,\text{ }\sigma _{a}=\sigma _{\tau }=0.02,%
\text{ }\sigma _{u}=0.
\end{eqnarray*}%
Generate $T=200$ artificial observations on the `endogenous' (in the sense
of Dynare) variables of the model. These are the variables in the `var'
list. The mod file provided, cggsim.mod, has 7 variables. Before doing the
simulation, you should add the growth rate of output to the equations of the
model and to the var list (call it `dy'.) That way, Dynare will also
simulate output growth. The variables simulated by Dynare are placed in the $%
n\times T$ matrix, oo\_.endo\_simul$.\footnote{%
Here, endo\_simul is the matrix, which is a `field' in the structure, oo\_.}$
The $n$ rows of oo\_.endo\_simul correspond to the $n=8$ variables in var, 
{\it listed in the order in which you have listed them in the var statement}
from the first to the last row. To verify the order that Dynare puts the
variables in, see how they are ordered in M\_.endo\_names in the
Dynare-created file, cggsim.m.

Retrieve output growth from oo\_.endo\_simul and get the log level of
output, $y,$ using $y=$cumsum($dysim$), where $dysim$ is the name I
arbitrarily assigned to the row of oo\_.endo\_simul corresponding to output
growth$.$ Also, retrieve $x$ from the appropriate row of oo\_.endo\_simul
and create natural output from the relation, $y^{\ast }=y-x.$

\begin{enumerate}
\item Compute the HP filter of $y$ with $\lambda =1$ and display a graph
with $y$ and $y^{T}.$ Do this also for $\lambda =1600$ and for $\lambda =$%
160,000,000. Do the results accord with what you would expect, given the
formula for the HP filter above?

\item Graph the HP filter trend, $y^{T},$ ($\lambda =1600)$ along with $y$
and $y^{\ast }.$ Note how actual output is somewhat more volatile than
potential or natural output (recall, the economy overreacts to technology
shocks). As a result, the HP filter with $\lambda =1600$ over smooths the
data. Graph $y_{t}-y_{t}^{T}$ and the true output gap, $x_{t},$ as well as $%
y,$ $y^{T}$ and $y^{\ast }.$ Compute the correlation between $%
y_{t}-y_{t}^{T} $ and $x_{t}$. Also compute the correlation for the case
where technology shocks are dominant (i.e., $\sigma _{a}=2,$ $\sigma _{\tau
}=0.02)$ and for the case where preference shocks are dominant (i.e., $%
\sigma _{a}=0.02,$ $\sigma _{\tau }=2).$ Interpret the results. The MATLAB
command for computing the correlation between two variables, $w_{t}$ and $%
u_{t},$ is corrcoef(w,u). The result of this calculation is a $2\times 2$
matrix with unity on the diagonal and the correlation on the off-diagonal.
\end{enumerate}

The model of this question lies close to the heart of the main paradigm
underlying the current view about the monetary transmission mechanism. Note
that in the case of this model, the hp-filter is not terrible as a guide to
the output gap. This is because the technology shock is the important shock
in the dynamics of the data, and the actual data overreact to the technology
shock. That is, the natural rate of output is a smooth version of the data.
Of course, this is only an example, and is something worth pursuing more
carefully using a DSGE model that has more solid empirical foundations.

\item Now we will do some estimation. First, we generate artificial data
from the baseline parameterization of the model. Place the simulated data,
oo\_.endo\_simul, in the matrix, data. Then, save these data to a MATLAB
file, data, using the instruction, save data data. Also, set periods = 5000
in the stoch\_simul command. Run the mod file using Dynare. This saves the
simulated data. Second, open cggest.mod.

\begin{enumerate}
\item First, do maximum likelihood estimation. Use 4,000 observations to
verify that everything is working properly. Consistency of maximum
likelihood implies that with this many observations, the probability that
the estimates are far from the true parameter values is low. Try doing the
estimation when you start far from the true parameter value, say with
rho=lambda=0.9. Despite the bad initial guess about the parameter values,
you should end up roughly at the true values.

\item Redo (a), but now with 30 observations, and you should see that
maximum likelihood still works well. Note that although the point estimates
look quite good, the standard error on lambda is rather large.
\end{enumerate}

\item Now do Bayesian estimation, using the inverted gamma distribution as
the prior on the two standard deviations and the beta distribution as the
prior on the two autocorrelations.

\begin{enumerate}
\item Set the mean of the priors over the parameters to the corresponding
true values. Set the standard deviation of the inverted gamma to 10 and of
the beta to 0.04. (It's hard to interpret these standard deviations
directly, but you will see graphs of the priors, which are easier to
interpret.) Use 30 observations in the estimation. Adjust the value of $k,$
so that you get a reasonable acceptance rate. I found that $k=1.2$ works
well. Have a look at the posteriors, and notice how, with one exception,
they are much tighter than the priors. The exception is lambda, where the
posterior and prior are very similar. This is evidence that there is little
information in the data about lambda.

\item Redo (a), but set the mean and standard deviation of the prior on
lambda equal to 0.95 and 0.04, respectively. Note how the prior and
posterior are again very similar. There is not much information in the data
about the value of lambda!

\item Note how the priors on $\sigma _{a}$ and $\rho $ have faint
`shoulders' on the right side. Redo (a), with $M=4,000$ ($M$ is mh\_replic,
which controls the number of MCMC replications). Note that the posteriors
are now smoother. Actually, $M=4,000$ is a small number of replications to
use in practice.

\item Now set the mean of the priors on the standard deviations to 0.1, far
from the truth. Set the prior standard deviation on the inverted gamma
distributions to $1$. Keep the observations at 30, and see how the
posteriors compare with the priors. (Reset $M=1,000$ so that the
computations go quickly.) Note that the posteriors move sharply back into
the neighborhood of 0.02. Evidently, there is a lot of information in the
data about these parameters.

\item Repeat (a) with 4,000 observations. Compare the priors and posteriors.
Note how, with one exception, the posteriors are `spikes'. The exception, of
course, is lambda. Still, the difference between the prior and posterior in
this case indicates there is information in the data about lambda.
\end{enumerate}

\item It is of interest to compare the posterior densities approximated by
the MCMC algorithm with the Laplace approximation. Consider the setup in 5
(a). You can recover all the information you need for these calculations
from the structure, oo\_. The posterior distributions of the parameters and
shock standard errors are in the structure oo\_.posterior\_density.
Posterior modes are in oo\_.posterior\_mode. Posterior standard deviations
(taken from the relevant diagonal parts of the inverse of the hessian of the
log criterion) appear in oo\_.posterior\_variance (my code for recovering
these objects is compareMCMCLaplace.m. Setting $M=10,000,$ I found%
\[
\]%
\[
\]%
\[
\FRAME{itbpF}{4.6512in}{3.8073in}{0in}{}{}{figcompare.eps}{\special{language
"Scientific Word";type "GRAPHIC";maintain-aspect-ratio TRUE;display
"USEDEF";valid_file "F";width 4.6512in;height 3.8073in;depth
0in;original-width 6.0848in;original-height 4.9741in;cropleft "0";croptop
"1";cropright "1";cropbottom "0";filename 'figcompare.eps';file-properties
"XNPEU";}} 
\]%
\[
\]%
When I set $M=100,000,$ the MCMC posteriors became smoother:%
\[
\]%
\[
\]%
\[
\FRAME{itbpF}{4.6512in}{3.8073in}{0in}{}{}{figcompare1.eps}{\special%
{language "Scientific Word";type "GRAPHIC";maintain-aspect-ratio
TRUE;display "USEDEF";valid_file "F";width 4.6512in;height 3.8073in;depth
0in;original-width 6.0848in;original-height 4.9741in;cropleft "0";croptop
"1";cropright "1";cropbottom "0";filename 'figcompare1.eps';file-properties
"XNPEU";}} 
\]%
\[
\]%
Note how much more similar the MCMC and Laplace posteriors are. The tail
areas of the MCMC posteriors have thinned out and now resemble more closely
the Laplace. Next, I set $M=1,000,000$ and obtained virtually the same
result as with $M=100,000:$%
\[
\]%
\[
\]%
\[
\FRAME{itbpF}{4.6512in}{3.8073in}{0in}{}{}{figcompare2.eps}{\special%
{language "Scientific Word";type "GRAPHIC";maintain-aspect-ratio
TRUE;display "USEDEF";valid_file "F";width 4.6512in;height 3.8073in;depth
0in;original-width 6.0848in;original-height 4.9741in;cropleft "0";croptop
"1";cropright "1";cropbottom "0";filename 'figcompare2.eps';file-properties
"XNPEU";}} 
\]%
\[
\]%
Thus, in this example it seems that the MCMC algorithm has roughly converged
for $M=100,000.$ In addition the Laplace and MCMC approximations deliver
very similar results, consistent with the conclusion that the Laplace
approach can used at the start and middle of a research project, while the
MCMC can be done later on. Note that in any particular project, you can
`test' this proposition doing comparison of the posterior distribution
obtained by the Laplace approximation with the posterior distribution
obtained by MCMC.

\item The output gap is not in the dataset used in the econometric
estimation. However, as noted in the handout, it is possible to use the
Kalman filter to estimate the output gap (actually, everything in the state)
from the available data. There are two ways to do this: `smoothing' uses the
entire data set and `filtering' only uses the part of the dataset prior to
the date for which the estimate of the gap is formed (thus, filtered data
are one-step-ahead forecasts). To activate the Kalman smoother in Dynare,
include the argument, smoother, in the estimation argument list. The
smoothed estimates will then be placed in a MATLAB structure
oo\_.SmoothedVariables. This structure can be accessed either directly from
the command line, or by selecting \TEXTsymbol{>}desktop\TEXTsymbol{>}%
workspace from the pull down menu available in the command window. You will
see that inside oo\_.SmoothedVariables there are a number of subcategories,
with output related to the Bayesian estimation (for example,
oo\_.SmoothedVariables.Median.x displays the median smoothed estimate of the
output gap, $x$). To see how well the Dynare-estimated version of the model
does at producing a good guess of the output gap, include the code,
analyzegap.m, at the end of your mod file. This shows you how to recover the
smoothed output gap from Dynare, and allows you to compare it with the
actual output gap, as well as with the hp-filtered estimate of the output
gap.

\item Dynare also reports confidence intervals for the smoothed variables
(e.g., oo\_.SmoothedVariables.HPDinf.x contains the upper bound of the 95
percent confidence interval for $x,$ in case you set conf\_sig =.95 in the
Dynare estimation command). These reflect parameter uncertainty, as well as
the difficulty of recovering these variables from the observed data when
they are not in the data set. If you run analyzegap.m down to line 41, you
will see what this confidence interval looks like, by comparison with the
actual gap. Note that occasionally, the actual gap lies outside the
confidence interval, as is to be expected.

\item The analysis in the previous question suggests that the output gap can
be estimated reliably using the estimated dsge model. However, in practice
one needs the output gap in real time. For this, the smoothed estimates of
the output gap are not a reliable indicator. Instead, it is useful to look
at the filtered estimates. These are found by running analyzegap.m to line
56 (the code shows how these are recovered from oo\_.FilteredVariables).
Note that there is a systematic phase shift between the estimated and actual
gaps. This is as expected. Turning points are hard to `see' in real time.
They become evident only after the fact. (Dynare also reports `updated'
variables. These are forecasts of the data based on current and past
observations. Not surprisingly, the updated `estimates' of variables that
happen to be in the econometrician's data set happen to coincide with their
true values. This is not so for filtered variables.)

\item It is interesting to see how the HP filter works in real time. By
running analyze.m down to line 75 one obtains an estimate of this. Note that
the HP filter does not exhibit the same phase shift as the filtered data.
This is because for date $t$ I have computed the HP filter using data up to
and including date $t.$

\item Dynare will also do forecasting. For this, one includes the argument,
forecast=xx, where xx indicates how many periods in the future you want to
forecast. (Put in xx=12.) To obtain the forecasts, as well as forecast
uncertainty, execute the rest of analyzegap.m. You can see from the
analyzegap.m code where in oo\_.PointForecast the forecasts as well as the
forecast uncertainty is stored.
\end{enumerate}

\end{document}
