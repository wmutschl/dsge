\documentclass{beamer} %%% F�R VORTRAG MIT PAUSEN
%\documentclass[handout]{beamer}  %%% F�R HANDOUT ALS PDF

\setbeamertemplate{navigation symbols}{}
\usetheme{Madrid}
\usecolortheme{seagull}
\beamersetuncovermixins{\opaqueness<1>{25}}{\opaqueness<2->{15}}
\usepackage[T1]{fontenc}
\usepackage{ae,aecompl}
\usepackage[ansinew]{inputenc}
\usepackage[ngerman]{babel}
\usepackage{amsmath}
\usepackage{amsfonts}
\usepackage[babel,german=quotes]{csquotes} %im deutschen �bliche Anf�hrungszeichen
\usepackage{verbatim}
\newcounter{saveenumi}
\newcommand{\seti}{\setcounter{saveenumi}{\value{enumi}}}
\newcommand{\conti}{\setcounter{enumi}{\value{saveenumi}}}


\begin{document}
\author[Willi Mutschler]{Dr. Andrea Beccarini $\qquad$ Willi Mutschler, M.Sc.}
\date{Summer 2012}
\institute[Institute of Econometrics]{Institute of Econometrics and Economic
Statistics\\University of M�nster\\willi.mutschler@uni-muenster.de}
\title{DSGE-Models}
\subtitle{Linearization and Solution Methods}

\begin{frame}
\titlepage
\end{frame}

\begin{frame}\frametitle{Linearization and Solution Methods}
\tableofcontents
\end{frame}

\section{Previously\dots}
\begin{frame}\frametitle{Previously\dots}\framesubtitle{}

\begin{itemize}
  \item Theory and intuition behind the Smets/Wouters' model as a prototype of current DSGE-models.
  \item Derivation of the structural form and log-linearization.
\begin{block}{Insight} A DSGE-model consists of a set of expected, nonlinear optimality conditions and transition equations for stochastic processes, which one has to solve.
\end{block}
\item A DSGE-model can in general be written as:
\begin{block}{General form}
        \begin{equation}\label{AllgForm}
            \boldsymbol{\Gamma}\left(E_t\mathbf{x_{t+1}}, \mathbf{x_t}, \boldsymbol{\upsilon_{t+1}}|\boldsymbol{\mu}\right)= \boldsymbol{\Gamma}\left(\mathbf{x_{t+1}}, \mathbf{x_t}, \boldsymbol{\upsilon_{t+1}},\boldsymbol{\eta_{t+1}}|\boldsymbol{\mu}\right)=0,
      \end{equation}
\end{block}
with $\mathbf{x_{t}}$: $(n \times 1)$-vector of stationary variables,
$\boldsymbol{\upsilon_t}$: ($m\times 1$)-vector of structural shocks,
$\boldsymbol{\mu}$: $(k \times 1)$-vector of parameters.
\end{itemize}
\end{frame}

\section{General form solution of a DSGE-model}
\begin{frame}\frametitle{General Form}
  \begin{itemize}
    \item Rational Expectations: $\boldsymbol{\eta_{t+1}} =
        E_t\mathbf{x_{t+1}} - \mathbf{x_{t+1}}$.
    \item Non-predictable expectation error
        $\boldsymbol{\eta_{t+1}}$ occurs due to the realization of structural shocks:
        $\boldsymbol{\eta_t}=f(\boldsymbol{\upsilon_t})$.
    \item $\mathbf{x_t}$ consist of $n_c$ control variables
        and $n_s$ state variables.
    \item Control variables are denoted by $\mathbf{c_t}$:
        optimal behavior of the agents as a function of the current state of the economy.
    \item State variables are denoted by $\mathbf{s_t}$. They consist of
        exogenous - independent of the decisions of the agents - and endogenous state variables,
        that can be influenced by the agents.
    \item $\mathbf{s_t}$ is a function of previous states and current shocks.
  \end{itemize}
\end{frame}


\begin{frame}\frametitle{Solution of a DSGE-model}\framesubtitle{}
\begin{itemize}
   \item To solve such a rational expectations model means to find so-called \emph{policy-functions} $c$ and $s$, that solve (at least approximately) the system of equations $\boldsymbol{\Gamma}$:
\begin{block}{Policy-functions} \centering
$\mathbf{c_t}=c(\mathbf{s_t}),\qquad \mathbf{s_t}=s(\mathbf{s_{t-1}},
\boldsymbol{\upsilon_t})$.
\end{block}
   \item  DSGE-models can be interpreted as \emph{state-space-models}.
   \item One distinguishes between linear and non-linear methods:
    \begin{itemize}
   \item Linear methods: Blanchard/Khan (1980), Binder and Pesaran
       (1997), Christiano (2002), King and Watson (1998), Klein
       (2000), Sims (2001) and Uhlig (1999).
   \item Frequently used nonlinear method: Schmitt-Groh�/Uribe
       (2004). Excellent overview: Heer and Maussner (2009).
   \end{itemize}
\end{itemize}
\end{frame}

\section{Repetition: Exercise 1}
\begin{frame}\frametitle{Repetition: Exercise 1}
Assume the following stochastic growth-model:
\begin{align*}
  \underset{c_t,k_{t+1}}{max} E_0 \sum_{t=0}^\infty \beta^t U(c_t) &\quad\text{with } U(c_t)=\frac{c_{t}^{1-\sigma}-1}{1-\sigma},\\
  c_t + k_{t+1} = z_tf(k_t) +(1-\delta)k_t &\quad \text{with } f(k_t)=k_t^\alpha, ~k_o \text{ given},\\
  log(z_t) = \rho log(z_{t-1}) + \varepsilon_{t} &\quad \text{with } \varepsilon_t \sim WN(0,\sigma_\varepsilon),~0<\rho<1.
\end{align*}
\begin{enumerate}[(a)]
   \item What is the unconditional expectation of $log(z_t)$, what of $z_t$? Find an expression for the unconditional variance of $log(z_t)$?
   \item Derive the first-order-conditions (FOC) by setting up the Bellman-equation and using methods of dynamic programming. What are state and control variables?
      \item Derive the first-order-conditions (FOC) by maximizing the lagrangian.
    \item Derive the \emph{steady-state} and linearize the
        FOC.
        \seti
   \end{enumerate}
\end{frame}

\section{Linear Solution Methods}
\begin{frame}\frametitle{Linear Solution Methods}%\framesubtitle{Einleitung}
\textbf{Pros}:
\begin{itemize}
   \item Simple linear state-space representation of the model, which in many cases is sufficiently exact.
   \item One can use the Kalman-filter to empirically evaluate the system.
\end{itemize}
\textbf{Cons}:
\begin{itemize}
   \item One looses important information during the linearization.
   \item Higher moments play an important role for analyzing markets, risk, welfare, etc.
   \item An approximation to, say, the second order can yield different results, because the variance of future shocks matters (risk premium).
\end{itemize}
\begin{block}{\emph{Certainty-equivalence-property}}
       In stochastic rational expectations models agents take the effect of future shocks into account. For a linearization to the first-order these expectations are zero, thus, they don't matter for the decision rules.
\end{block}

\end{frame}

\begin{frame}\frametitle{Linear Solution Methods}\framesubtitle{}
\begin{itemize}
   \item First linearize or log-linearize the general form \eqref{AllgForm} around the
       deterministic \emph{steady-state}.
   \item Together with the transition equation of the stochastic processes one gets the reduced-form model:
\begin{align*}
    \mathbf{A} \mathbf{x_{t+1}} =
\mathbf{B} \mathbf{x_t} + \mathbf{C} \boldsymbol{\upsilon_{t+1}} + \mathbf{D}
\boldsymbol{\eta_{t+1}} + \mathbf{E}.
\end{align*}
\item The matrices $\mathbf{A}, \mathbf{B}, \mathbf{C}$ and $\mathbf{D}$
    are functions of the structural parameters
    $\boldsymbol{\mu}$; $\mathbf{E}$ is a vector of constants (mostly
    zero).
\item The solution of this linear representation has a VAR(1)-form:
\begin{align}\label{Linlsg}
\mathbf{x_{t+1}} = \mathbf{F(\boldsymbol{\mu})} \mathbf{x_t} + \mathbf{G(\boldsymbol{\mu})} \boldsymbol{\upsilon_{t+1}},
\end{align}
with $\mathbf{F}$ and $\mathbf{G}$ being functions of the parameters
$\boldsymbol{\mu}$.
\item The motor of the model is the vector of exogenous shocks
    $\boldsymbol{\upsilon_t}$.
\item \eqref{Linlsg} describes thus, the fluctuations around the steady-state as well as the decision rules given the stochastic innovations.
\end{itemize}
\end{frame}

\subsection{The Sims (2001)-algorithm}
\begin{frame}\frametitle{The Sims (2001)-algorithm}\framesubtitle{Concepts}
\begin{itemize}
\begin{block}{Notation}
Predetermined variables: $\qquad ~E_t X_{1,t+1}=X_{1,t+1}$,\\ Non-predetermined variables: $E_t X_{2,t+1}=E_t X_{2,t+1}$.
\end{block}
\begin{block}{Unitary matrices}
    $\mathbf{M}'\mathbf{M}=\mathbf{M}\mathbf{M}'=\mathbf{I}$ are the complex analogous to orthogonal matrices. They are diagonalizable.
\end{block}
   \item The method of Sims (2001) begins with a
       \emph{QZ-factorization} (General Schur decomposition), in which the matrices $\mathbf{A}$ and
       $\mathbf{B}$ are transformed into unitary upper triangular matrices:
\begin{align*}
 \mathbf{A} = \mathbf{Q}' \boldsymbol{\Lambda} \mathbf{Z}' ,\qquad \mathbf{B} = \mathbf{Q}' \boldsymbol{\Omega} \mathbf{Z}'.
\end{align*}
\item $\boldsymbol{\Lambda}$ and $\boldsymbol{\Omega}$ are upper triangular matrices with the generalized Eigenvalues of
    $\mathbf{A}$ and $\mathbf{B}$, and they are sorted in increasing order from left to right.
\end{itemize}
\end{frame}

\begin{frame}\frametitle{The Sims (2001)-algorithm}\framesubtitle{}
\begin{itemize}
   \item The Eigenvalues determine if a system of equations converges or explodes.
\begin{block}{Blanchard/Khan-conditions}
The number of Eigenvalues, that are in absolute terms greater than 1, must be equal to the number of non-predetermined variables, in order to get a stable solution (saddle-path).
\end{block}
\item Let $\mathbf{z_{t+1}}=\mathbf{Z}'\mathbf{x_{t+1}}$, then the system can be divided into a non-explosive (upper) and an explosive part (lower):
\begin{align}\label{sims}
  \begin{bmatrix} \boldsymbol{\Lambda_{11}} & \boldsymbol{\Lambda_{12}}\\ \mathbf{0} & \boldsymbol{\Lambda_{22}} \end{bmatrix}
  \begin{pmatrix} \mathbf{z_{1,t+1}} \\ \mathbf{z_{2,t+1}} \end{pmatrix}
= \begin{bmatrix} \boldsymbol{\Omega_{11}}&\boldsymbol{\Omega_{12}}\\\mathbf{0}&\boldsymbol{\Omega_{22}}\end{bmatrix}
  \begin{pmatrix} \mathbf{z_{1,t}} \\ \mathbf{z_{2,t}} \end{pmatrix}\\
+ \begin{pmatrix} \mathbf{Q_1 }\\ \mathbf{Q_2} \end{pmatrix}
  \begin{bmatrix} \mathbf{E_1}+\mathbf{C_1}\boldsymbol{\upsilon_{1,t+1}} + \mathbf{D_1} \boldsymbol{\eta_{1,t+1}} \\ \mathbf{E_2}+\mathbf{C_2}\boldsymbol{\upsilon_{2,t+1}} +\mathbf{ D_2} \boldsymbol{\eta_{2,t+1}} \end{bmatrix}\nonumber.
\end{align}

\end{itemize}
\end{frame}

\begin{frame}\frametitle{The Sims (2001)-Algorithmus}\framesubtitle{}
\begin{itemize}
   \item The difference equations belonging to the Eigenvalues greater than 1 are solved forwards.
   \item Remark: $\underset{t\rightarrow
       \infty}{\lim}\left(\boldsymbol{\Omega_{22}}^{-1}
       \boldsymbol{\Lambda_{22}}\right)^t \mathbf{z_{2,t}}=\mathbf{0}$
       and for all $s>0: E_t \boldsymbol{\upsilon_{2,t+s}} = E_t
\boldsymbol{\eta_{2,t+s}}=0$ (Expectations don't matter)
\begin{align*}
  \mathbf{z_{2,t}} & = \boldsymbol{\Omega_{22}}^{-1} \boldsymbol{\Lambda_{22}} \mathbf{z_{2,t+1}} - \boldsymbol{\Omega_{22}}^{-1}\mathbf{ Q_2} \left[\mathbf{E_2} + \mathbf{C_2} \boldsymbol{\upsilon_{2,t+1}} + \mathbf{D_2} \boldsymbol{\eta_{2,t+1}}\right]\\
  & = -\sum_{i=0}^\infty \left(\boldsymbol{\Omega_{22}}^{-1} \boldsymbol{\Lambda_{22}}\right)^i \boldsymbol{\Omega_{22}}^{-1} \mathbf{Q_2 }\left[\mathbf{E_2} + \mathbf{C_2} \boldsymbol{\upsilon_{2,t+1+i}} + \mathbf{D_2} \boldsymbol{\eta_{2,t+1+i}}\right]\\
  & = -\sum_{i=0}^\infty \left(\boldsymbol{\Omega_{22}}^{-1} \boldsymbol{\Lambda_{22}}\right)^i \boldsymbol{\Omega_{22}}^{-1} \mathbf{Q_2} \mathbf{E_2}\\
  & = -\left[\mathbf{I}-\boldsymbol{\Omega_{22}}^{-1}\boldsymbol{\Lambda_{22}}\right]^{-1} \boldsymbol{\Omega_{22}}^{-1} \mathbf{Q_2} \mathbf{E_2}
   = \left[\boldsymbol{\Lambda_{22}} - \boldsymbol{\Omega_{22}}\right]^{-1} \mathbf{Q_2} \mathbf{E_2}.
\end{align*}
\end{itemize}
\end{frame}

\begin{frame}\frametitle{The Sims (2001)-algorithm}\framesubtitle{}
\begin{itemize}
   \item The difference equations belonging to the Eigenvalues less or equal than 1 are solved backwards.
   \item Remark: Systematical relationship betweend the expectation errors
       $\boldsymbol{\eta_{1,t}} \& \boldsymbol{\eta_{2,t}}$.
\begin{block}{Sufficient condition for a stable saddle-path}\centering $\mathbf{Q_1}\mathbf{D} =
  \boldsymbol{\Phi} \mathbf{Q_2} \mathbf{D}$.
\end{block}
\item $\boldsymbol{\Phi}$ is of dimension $n_s \times n_c$ (with
    $\mathbf{z_{1,t}}: n_s\times1$ and $\mathbf{z_{2,t}}: n_c \times 1$).
\item \eqref{sims} can be rewritten as:
\begin{multline*}
\underbrace{\begin{bmatrix} \underset{n_s \times n_s}{\mathbf{I}} &
\underset{n_s \times n_c}{-\boldsymbol{\Phi}}\end{bmatrix}}_{n_s \times
(n_s+n_c)}
  \begin{bmatrix}\boldsymbol{\Lambda_{11}} & \boldsymbol{\Lambda_{12}}\\\mathbf{0}&\boldsymbol{\Lambda_{22}}\end{bmatrix}
  \begin{pmatrix} \mathbf{z_{1,t}} \\ \mathbf{z_{2,t}}\end{pmatrix} =\\
  \begin{bmatrix} \mathbf{I} & -\boldsymbol{\Phi}\end{bmatrix} \begin{bmatrix} \boldsymbol{\Omega_{11}}&\boldsymbol{\Omega_{12}}\\\mathbf{0}&\boldsymbol{\Omega_{22}}\end{bmatrix}
  \begin{pmatrix} \mathbf{z_{1,t-1}} \\ \mathbf{z_{2,t-1}} \end{pmatrix}
+ \begin{bmatrix} \mathbf{I} & -\boldsymbol{\Phi}\end{bmatrix}
\begin{pmatrix} \mathbf{Q_1} \\ \mathbf{Q_2} \end{pmatrix}
  \begin{bmatrix} \mathbf{E}+\mathbf{C}\boldsymbol{\upsilon_{t}} + \mathbf{D} \boldsymbol{\eta_{t}} \end{bmatrix}.
\end{multline*}
\end{itemize}
\end{frame}

\begin{frame}\frametitle{The Sims (2001)-algorithm}\framesubtitle{}
\begin{multline*} \Leftrightarrow \begin{bmatrix}
       \boldsymbol{\Lambda_{11}} &
       \boldsymbol{\Lambda_{12}}-\boldsymbol{\Phi}
       \boldsymbol{\Lambda_{22}} \end{bmatrix}
  \begin{pmatrix} \mathbf{z_{1,t}} \\ \mathbf{z_{2,t}}\end{pmatrix} =
  \begin{bmatrix} \boldsymbol{\Omega_{11}} & \boldsymbol{\Omega_{12}}-\boldsymbol{\Phi} \boldsymbol{\Omega_{22}} \end{bmatrix}
  \begin{pmatrix} \mathbf{z_{1,t-1}} \\ \mathbf{z_{2,t-1}}\end{pmatrix}\\
+ \begin{bmatrix}\mathbf{ Q_1 }- \boldsymbol{\Phi} \mathbf{Q_2}
\end{bmatrix}
  \begin{bmatrix} \mathbf{E}+\mathbf{C}\boldsymbol{\upsilon_{t}} \end{bmatrix}
+ \underbrace{\begin{bmatrix} \left(\mathbf{Q_1} - \boldsymbol{\Phi}
\mathbf{Q_2}\right) \mathbf{D}
\boldsymbol{\eta_{t}}\end{bmatrix}}_{=\mathbf{0}}.
\end{multline*}
\begin{block}{The algorithm}\centering
QZ-factorization: $\mathbf{A} = \mathbf{Q}' \boldsymbol{\Lambda} \mathbf{Z}'
, \mathbf{B} = \mathbf{Q}' \boldsymbol{\Omega} \mathbf{Z}'$,
$\mathbf{x_{t}}=\mathbf{Z}
\begin{pmatrix}\mathbf{z_{1,t}}\\\mathbf{z_{2,t}}\end{pmatrix}$,
\begin{align*}
\begin{split}
  \mathbf{z_{1,t} } =& - \boldsymbol{\Lambda_{11}}^{-1}\left(\boldsymbol{\Lambda_{12}}-\boldsymbol{\Phi} \boldsymbol{\Lambda_{22}}\right) \mathbf{z_{2,t}}
  + \boldsymbol{\Lambda_{11}}^{-1}
  \boldsymbol{\Omega_{11}}\mathbf{z_{1,t-1}} \\
  &+\boldsymbol{\Lambda_{11}}^{-1}\left(\boldsymbol{\Omega_{12}}-\boldsymbol{\Phi}
  \boldsymbol{\Omega_{22}}\right)\mathbf{z_{2,t-1}} +
  \boldsymbol{\Lambda_{11}}^{-1} \left(\mathbf{Q_1} - \boldsymbol{\Phi}
  \mathbf{Q_2}\right)\left(\mathbf{E}+\mathbf{C}
  \boldsymbol{\upsilon_t}\right),
\end{split}\\
   \mathbf{ z_{2,t}} =& \left(\boldsymbol{\Lambda_{22}}-\boldsymbol{\Omega_{22}}\right)^{-1}\mathbf{Q_2} \mathbf{E}.
\end{align*}
\end{block}
\end{frame}

\subsection{Exercise 1: continued}
\begin{frame}\frametitle{Exercise 1: continued}\framesubtitle{}
The Sims (2001)-algorithm is available for different program packages, and it is implemented in Dynare.
\begin{enumerate}[(a)]\conti
   \item Rewrite the linearized model into the form $ \mathbf{A} \mathbf{x_{t+1}} =
       \mathbf{B} \mathbf{x_t} + \mathbf{C}
\boldsymbol{\upsilon_{t+1}} + \mathbf{D} \boldsymbol{\eta_{t+1}} +
\mathbf{E} $
\item Solve for the \emph{policy-function} by using the Sims (2001)-
    Algorithm. Assume the following values for the parameters:
    $\beta=0.99,\quad \alpha=0.36,\quad \sigma=2,\quad \delta
    =0.025,\quad \rho=0.9$.
\end{enumerate}
\end{frame}

\begin{frame}\frametitle{Homework}
\begin{itemize}
\item Install the newest version of Matlab.
\item https://zivdav.uni-muenster.de/ddfs/Soft.ZIV/TheMathWorks/ .
\item Because of license issues you have to be connected to the university either via WLAN or VPN.
\item Download all files from
    http://sims.princeton.edu/yftp/gensys/mfiles/ into a folder called gensys.
    Make this folder available to Matlab (\texttt{File-SetPath-Add Folder}).
\item Install the newest version of Dynare (http://www.dynare.org/). Make the Dynare folder $c:\backslash dynare  \backslash aktuelle~Version  \backslash matlab)$\\ available to Matlab (\texttt{File-SetPath-Add Folder}).
\end{itemize}
\end{frame}

\end{document}
