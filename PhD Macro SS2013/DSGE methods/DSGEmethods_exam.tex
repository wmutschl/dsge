\documentclass[10pt,a4paper]{article}
\usepackage[english]{babel}
\usepackage{csquotes}
\usepackage{setspace}
\onehalfspacing
\usepackage[top=30mm,left=30mm,bottom=30mm]{geometry}
\usepackage[automark]{scrpage2}

\usepackage{hyperref}
\hypersetup{
	pdfauthor={Willi Mutschler},
	pdftitle={},
	pdfproducer={LaTeX}, % Software mit der das ganze Produziert wurde...
	colorlinks=false,   % macht die Schrift von Hyperlinks rot (true), false = schwarze Schrift
    pdfborder={0 0 0}, % keine Rechteckmarkierung um Links
  %pdfpagemode=None,  % PDF-Viewer startet ohne Inhaltsverzeichnis et.al.
  pdfstartview={FitH}, % PDF-Viewer benutzt beim Start bestimmte Seitenbreite
  plainpages = false, % Damit auch zwischen r\"{o}mischen und arabischen Zahlen unterschieden werden kann
}
\usepackage{amssymb,amsmath}
\usepackage{xcolor}
\usepackage{enumerate}
\usepackage[hyperref=true,style=authoryear, dashed=false, maxnames=3,backend=bibtex8,doi=false,isbn=false,backref=true]{biblatex}
\usepackage[affil-it]{authblk}
\makeatletter
\def\@maketitle{%
  \newpage
  \null
  \vskip 2em%
  \begin{center}%
  \let \footnote \thanks
    {\Large\bfseries \@title \par}%
    \vskip 1.5em%
    {\normalsize
      \lineskip .5em%
      \begin{tabular}[t]{c}%
        \@author
      \end{tabular}\par}%
    \vskip 1em%
    {\normalsize \@date}%
  \end{center}%
  \par
  \vskip 1.5em}
\makeatother


\begin{document}
\title{PhD Macroeconomics -- DSGE methods\\ Summer 2013\\ Examination}
\author{Willi Mutschler%
  \thanks{Electronic address: \texttt{willi.mutschler@uni-muenster.de}}}
\affil{Institute for Econometrics and Economic Statistics,\\ University of M\"unster, Germany}
\date{Due date: August 31, 2013}
\maketitle
\thispagestyle{empty}

\renewcommand\abstractname{~}
\begin{abstract}~
\begin{itemize}
\item Answer \textbf{all} of the following three exercises.
\item Hand in your solutions before August 31, 2012.
\item Please e-mail the solutions to willi.mutschler@uni-muenster.de
\item The solution files should contain your Dynare code and all
additional files \newline
(preferably \texttt{pdf}, not \texttt{doc} or \texttt{docx}).
\item \textbf{Important}: Please indicate if you are a Master student or a
PhD student. If you are a Master student, please also give your registration
number.
\item All students must work on their own.
\item If there are any questions, contact Willi Mutschler in room 305 or via e-mail.
\end{itemize}
\end{abstract}

\begin{center}\Huge{GOOD LUCK!}\end{center}\normalsize
\newpage \setcounter{page}{1}
\section{Dynare}
Consider the Clarida-Gali-Gertler model from the lecture. (Hint: Use the cggexam.mod as starting point for your analysis)
\begin{enumerate}
  \item Replace the time series representation of $a_t$ with $$a_t = \rho a_{t-1} + \varepsilon_{t}^a$$ Consider the response of the economy to a technology shock $\varepsilon_t^a$. Does the economy over- or under-respond to the shock relative to its \emph{natural} response? How does this compare to the unit root case? Give also economic intuition for the response. Hint: Use the Matlab function plots.m to get some prettier plots (put everything in one folder and write plots as the last command of your mod-file).
  \item Now assume that agents have advance information (news) about the future realization of the technology shock, i.e.
  $$ a_t = \rho a_{t-1} + \xi_t^0 + \xi_{t-1}^1$$ where $\xi_t^0$ and $\xi_t^1$ are both iid. In economic terms agents see $\xi_t^0$ at time $t$ and they see $\xi_{t-1}^0$ at time $t-1$. Introduce this into your code and consider a news shock to $\xi_t^1$. What happens with inflation and the output gap? Provide intuition behind this apparently contradictory result.
  \item What happens with the response of the economy due to a news shock relative to its natural response, if the natural rate of interest is introduced into the policy rule, i.e.
   \begin{equation*} r_{t} = r_t^{\ast} + \alpha (r_{t-1}-r^\ast_{t-1})+(1-\alpha )\left[ \phi _{\pi }\pi _{t}+\phi _{x}x_{t}\right]\end{equation*}
\end{enumerate}
\newpage
\section{Solution methods}
Consider a simplified Asset-Pricing Model given by
\begin{align*}
u'(d_t) \cdot p_t &= \beta E_t \left[u'(d_{t+1})\cdot(p_{t+1}+d_{t+1})\right]\\
d_{t+1} & = \rho d_t + e_{t+1}, \qquad e_{t+1} \sim N(0,\sigma_e^2)
\end{align*}
in which $u'(d_t)$ denotes the first order derivative of the utility function $u(d_t)$, $\beta$ the discount factor, $\rho$ a time persistence parameter, $p_t$ the price and $d_t$ the dividend of a single asset in the environment with steady states $\overline{p}$ and $\overline{d}$. $e_t$ is a gaussian random variable with zero mean and standard deviation $\sigma_e$. 
\begin{enumerate}
  \item In the notation of the lecture, which variable is the state $(x_t)$ and which the control $(y_t)$ variable?
  \item Derive the decision rule for the state variable in the form of
  \begin{align*}
    x_{t+1} = \overline{x} + h(x_t,\sigma) + \sigma \varepsilon_{t+1}
  \end{align*}
  How would you set $\sigma$ in this case?
  \item The decision rule for the control variable up to second order is of the form
  \begin{align*}
    y_{t+1} = \overline{y} + [g_x](x_t - \overline{x}) + [g_\sigma]\sigma + \frac{1}{2}[g_{xx}](x_t-\overline{x})^2 + \frac{1}{2}[g_{x\sigma}](x_t-\overline{x})\sigma + \frac{1}{2}[g_{\sigma x}] \sigma (x_t-\overline{x}) + \frac{1}{2}[g_{\sigma \sigma}] \sigma^2
  \end{align*}
  Derive the expressions for $\overline{y}, [g_x], [g_{xx}], [g_{x \sigma}], [g_{\sigma x}]$ analytically (you don't have to do it for $g_{\sigma\sigma}$). \\
  \textbf{Hints:}
  \begin{itemize}
    \item Set up $F(x_t,\sigma)=0$ and differentiate with respect to $x_t$ and $\sigma$ twice, THEN take expectation of $F_{x}=0, F_\sigma=0, F_{xx}=0$ and evaluate at $(x_t,\sigma)=(\overline{x},0)$.
    \item For $g_x$ and $g_{xx}$ you should get:
  \begin{align*}
  g_x &= \frac{-u''(\overline{d})\cdot\overline{y} + \beta \rho u''(\overline{d})\cdot(\overline{y}+\overline{x})+\beta\rho u'(\overline{d})}{u'(\overline{d})\cdot(1-\rho \beta)},\\
  g_{xx} &= \frac{-u'''(\overline{d})\cdot \overline{y}-2 u''(\overline{d})\cdot g_x + \beta \rho^2 u'''(\overline{d})\cdot(\overline{y}+\overline{x})+2\beta\rho^2u''(\overline{d})\cdot(\overline{y}+1)}{u'(\overline{d})(1-\beta\rho^2)}
  \end{align*}
  \end{itemize}
\end{enumerate}
\newpage
\section{Estimation methods}
Consider the following simplified RBC-model (social planer problem);
\begin{align*}
  \underset{\{c_{t+j},l_{t+j},k_{t+j}\}_{j=0}^\infty}{max} W_t &= \sum_{j=0}^\infty \beta^j u(c_{t+j},l_{t+j})\\
  s.t.\quad y_t &= c_t + i_t, & A_t &= A e^{a_t}, \\
  y_t &= A_t f(k_{t-1},l_t), & a_t &= \rho a_{t-1} + \varepsilon_t,\\
  k_t &= i_t +(1-\delta)k_{t-1}, &  \varepsilon_t &\sim N(0,\sigma_{\varepsilon}^2),
\end{align*}
where preferences and technology follow:
\begin{align*}
  u(c_t,l_t)= \frac{\left[c_t^\theta (1-l_t)^{1-\theta}\right]^{1-\tau}}{1-\tau}, \qquad f(k_{t-1},l_t)=\left[\alpha k_{t-1}^\psi + (1-\alpha)l_t^\psi\right]^{1/\psi}.
\end{align*}
Optimality is given by:
\begin{eqnarray*}
 &u_c(c_t,l_t) - \beta E_t \left\{u_c(c_{t+1},l_{t+1}) \left[A_{t+1} f_k(k_t, l_{t+1})+1-\delta \right]\right\} = 0,\\
 & -\frac{u_l(c_t,l_t)}{u_c(c_t,l_t)}-A_t f_l(k_{t-1},l_t)=0,\\
 & c_t + k_t -A_t f(k_{t-1},l_t) -(1-\delta)k_{t-1}=0.
\end{eqnarray*}
See rbcestim.mod for this model and the Bayesian estimation we did in the lecture. \textbf{RUN IT ONCE TO GET SIMULATED DATA AND ESTIMATION.}\\  Now we will use the same dataset to estimate the parameters of a misspecified model. We will use the same model, however, with a small difference, i.e. technology follows a Cobb-Douglas production function. See rbcexam.mod for the new model equations and steady-state block.
\begin{enumerate}
  \item Define priors for $\alpha,\theta$ and $\tau$. How many observable variables do you need? Choose an appropriate number of observables.
  \item Estimate the posterior mode using the \texttt{estimation} command and a limited sample with 200 observations. Check the posterior mode using \texttt{mode\_check}. If you get errors due to a non-positive definite Hessian, try a different optimization algorithm or change the initial values.
  \item If you are satisfied with the posterior mode, approximate the posterior distribution using the Metropolis-Hastings-Algorithm with $3\times 5000$ iterations. How large is your acceptance rate (change $jh\_scale$ if you're not satisified). Provide also the diagnostic plots. If the algorithm does not converge to the (ergodic) posterior-distribution, repeat the algorithm with 1000 more iterations without discarding the previous draws.
  \item Take a stand on your Bayesian estimation. What is good, what can be improved and how? Compare the estimation of the common parameters of the true model with the misspecified model. 
  \item Calculate the \emph{posterior-odds} and the \emph{posterior-model-probabilities} weighting each model by 0.5 a prior. Hint: Write a mod-file with the following code and execute it:
\begin{verbatim}
model_comparison rbcestim(0.5) rbcexam(0.5);
\end{verbatim}
\end{enumerate}

\end{document}
