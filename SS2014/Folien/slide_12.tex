\documentclass{beamer} %%% FÜR VORTRAG MIT PAUSEN
%\documentclass[handout]{beamer}  %%% FÜR HANDOUT ALS PDF

\setbeamertemplate{navigation symbols}{}
\usetheme{Madrid}
\usecolortheme{seagull}
\beamersetuncovermixins{\opaqueness<1>{25}}{\opaqueness<2->{15}}
\usepackage[T1]{fontenc}
\usepackage{ae,aecompl}
\usepackage[ansinew]{inputenc}
\usepackage[ngerman]{babel}
\usepackage{amsmath}
\usepackage{amsfonts}
\usepackage[babel,german=quotes]{csquotes} %im deutschen übliche Anführungszeichen
\usepackage{verbatim}
\newcounter{saveenumi}
\newcommand{\seti}{\setcounter{saveenumi}{\value{enumi}}}
\newcommand{\conti}{\setcounter{enumi}{\value{saveenumi}}}
\usepackage{verbatim}

\begin{document}
\author[Willi Mutschler]{Willi Mutschler, M.Sc.}
\date{Summer 2014}
\institute[Institute of Econometrics]{Institute of Econometrics and Economic
Statistics\\University of M�nster\\willi.mutschler@uni-muenster.de}
\title{DSGE Methods}
\subtitle{Calibration of DSGE models}

\begin{frame}
\titlepage
\end{frame}

\section{Overview Estimation-Methods}
\begin{frame}\frametitle{Overview Estimation-Methods}\framesubtitle{}
\begin{itemize}
    \item Econometrically, a DSGE-Model is a state-space model of which
        one has to determine the parameters.
      \item Three concepts:
      \begin{enumerate}
        \item \textbf{Calibration}: The parameters are set in such a
            way, that they closely correspond to some theoretical
            moment or stylized fact of data.
        \item \textbf{Methods of limited information} or weak
            econometric interpretation: Minimize the distance between
            theoretical and empirical moments, i.e.
            \emph{General-Method-of-Moments} or \emph{Indirect
            Inference}.
        \item \textbf{Methods of full information} or strong
            econometric interpretation: The goal is an exact
            characterization of observed data, i.e.
            \emph{Maximum-Likelihood} or \emph{Bayesian methods}.
      \end{enumerate}
\end{itemize}
\end{frame}

\section{Calibration}
\begin{frame}\frametitle{Calibration}\framesubtitle{}
\begin{itemize}
   \item Goal: To answer a specific quantitative research question using
       a structural model.
   \item Construct and parameterize the model such, that it corresponds
       to certain properties of the true economy.
   \item Use steady-state-characteristics for choosing the parameters in
       accordance with observed data.
   \item Often: stable long-run averages (wages, working-hours, interest
       rates, inflation, consumption-shares, government-spending-ratios,
       etc.).
   \item You can use micro-studies as well, however, one has to be
       careful about the aggregation!
\end{itemize}
\end{frame}

\subsection{Hints for calibrating a model}
\begin{frame}\frametitle{Calibration}\framesubtitle{Hints for calibrating a model}

\begin{itemize}
   \item Use long-term averages of interest rates, inflation, average
       growth of productivity, etc. for \emph{steady-state} values.
   \item BUT: Weil (1989) shows, that in models with representative
       agents there is an overestimation of \emph{steady-state} interest
       rates (\emph{risk-free rate puzzle}). It is possible that you get
       absurd constellation of parameters, like a discount-factor of
       $\beta>1$.
   \item Usual mark-up on prices is around 1.15 (Corsetti et al (2012)).
   \item Intertemporal elasticity of substitution  $1/\sigma$ somewhere
       between $\sigma=1$ and $\sigma=3$ (King, Plosser and Rebelo
       (1988), Rotemberg and Woodford (1992), Lucas (2003)).
\end{itemize}
\end{frame}

\begin{frame}\frametitle{Calibration}\framesubtitle{Hints for calibrating a model}
\begin{itemize}
   \item Rigidity of prices: For an average price adjustment of 12-15
       months see Keen and Wang (2007).
   \item Coefficients of monetary policy: Often Taylor-Rule, you can use
       the relative coefficients to put more emphasize/weight on the
       stability of prices or on smoothing the business cycle.
   \item Parameters of stochastic processes: Often persistent, small
       standard-deviations, otherwise you get high oscillations. You can
       also estimate the production function via OLS (Solow-residual).
   \item How to choose shocks: Look at similar studies: Christiano,
       Eichenbaum and Evans (2005), Smets and Wouters (2003), etc..
   \item Ultimately: Try \& Error!
\end{itemize}
\end{frame}


\begin{frame}\frametitle{Exercise:}\framesubtitle{Calibration of a RBC-model with monopolistic competition}
The structural form of a DSGE-model is given by the following equations
       \begin{align}
         \frac{1}{c_t} &= \beta E_t \left[\frac{1}{c_{t+1}} (1+r_{t+1}-\delta)\right]\\
         w_t &= \psi \frac{c_t}{1-l_t}\\
         y_t &= c_t + i_t\\
         y_t &= A_t k_{t}^\alpha l_t^{1-\alpha}\\
         w_t & = (1-\alpha) \frac{y_t}{l_t} \frac{\varepsilon -1}{\varepsilon}\\
         r_t & = \alpha \frac{y_t}{k_t} \frac{\varepsilon -1}{\varepsilon}\\
         i_t &= k_{t+1} - (1-\delta) k_t\\
         log(A_t) &= \rho log(A_{t-1}) + \epsilon_t
       \end{align}
\end{frame}


\begin{frame}\frametitle{Exercise:}\framesubtitle{Calibration of a RBC-model with monopolistic competition}
\begin{enumerate}[(a)]
   \item Interpret the equations. What are state variables, what are control variables, what are the parameters of the model?
   \item Write a mod-file for the model and calibrate the vector of
       parameters. Simulate the model for 1000 periods with Dynare. Plot the path of consumption for 100 periods.
\end{enumerate}
\end{frame}

\section{Calibration - Pros \& Cons}
\begin{frame}\frametitle{Calibration}\framesubtitle{Pros}
\begin{itemize}
   \item Calibration is commonly used in the literature. It gives a first
       impression, a flavor of the strengths and weaknesses of a model.
   \item A good calibration can provide a valuable and precise image of
       data.
   \item Using different calibrations, one can asses interesting
       implications of different policies:
       \begin{itemize}
       \item How does the economy react, if the central bank focuses
           more on smoothing the business cycle than on price
           stability?
           \item What happens to consumption, if the households have
               a strong intertemporal elasticity of substitution?
               What if it is low?
       \end{itemize}
\end{itemize}
\end{frame}

\begin{frame}\frametitle{Calibration}\framesubtitle{Cons}
\begin{itemize}
   \item This Ad-hoc-approach is at the center of criticism of
       DSGE-models.
   \item There is no statistical foundation, it is based upon subjective
       views, assessments and opinions.
   \item Many parameter, such as those of the exogenous processes, leave
       room for different values and interpretations (intertemporal
       elasticity of substitution, monetary and fiscal parameters,
       coefficients of rigidity, standard deviations, etc.).
\end{itemize}
\begin{block}{Prescott (1986, S.~10) regarding RBC-models:} The models constructed within this theoretical framework are
necessarily \textbf{highly abstract}. Consequently, they are necessarily
false, and statistical hypothesis testing will reject them. This does not
imply, however, that nothing can be learned from such a \textbf{quantitative
theoretical exercise}.\end{block}
\end{frame}

\end{document}
